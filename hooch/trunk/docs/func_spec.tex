% $Id: func_spec.tex,v 1.7 2003/12/19 20:53:11 airhead Exp $
%
% Hooch Functional Specification
%

\documentclass[a4paper]{book}

\usepackage[english]{babel}
\usepackage{url}
\usepackage{fancyhdr}
\usepackage{hyperref}

\title{Hooch Functional Specification}
\date{October 2003}
\author{Peter~Bex\and{}Vincent~Driessen}
\begin{document}

\frontmatter

\maketitle
\tableofcontents


\mainmatter

\chapter{Overview}

Hooch is an e-mail client for Unix-like operating systems.

This document will only discuss the general working of the program, not
technical implementation details. For these, see the Hooch Technical
Specification.


\section{Major features}

Hooch supports the following major features:
\begin{itemize}
\item {\bf Person-based address book.} The address book consists of persons,
	mailing lists and groups of persons. This way, a person can have
	multiple addresses while a list cannot.
\item {\bf Class-based configuration file.} The configuration is done in
	classes, making the overall design more modular (modules specify
	their own classes) and the config files very human-readable.
\item {\bf Mailing list handling.} Mailing lists are supported by providing
	thread-based views for messages and replying to the list instead of
	the sender of a message.
\item {\bf Multiple user interfaces.} The design is modular, allowing for
	multiple UIs. The current aim is to create both a graphical one and
	a curses-based one. The latter one is initially the most important.
\item {\bf Maximum configurability.} The user can provide their own
	keymappings, color schemes and lay-out, depending on the selected
	interface. Modules supply their own configurable parameters.
\item {\bf Integration of major external programs by plugins.} PGP, for example
	is supported by presenting a list of available PGP keys and options for
	encrypting and signing. The actual PGP encryption is performed by
	external tools. This is done by a crypto module which in turn has a
	PGP module.
\end{itemize}


\section{Unsupported features}

Hooch will not support the following features:
\begin{itemize}
\item {\bf Retrieving mail from a server.} This should be delegated to
	another application, for example fetchmail.
\item {\bf Filtering mail.} This is also delegated. Hooch assumes the mail
	to be filtered when it reads it.
\item {\bf Composing mail.} The user can specify their favorite editor to
	type their mail with.
\item {\bf Post-processing mail.} Encryption of mail, for example, can be
	achieved by piping the e-mail through a PGP program.
\item {\bf Reading mime attachments.} Any non-plaintext attachments should
	be read using an external program. There \emph{are} provisions for
	automatically viewing attachments of specified types with specified
	programs.
\end{itemize}


\section{Typical usage}

What follows is a typical example of day-to-day usage of Hooch.

Jack logs in to his system and starts up Hooch. He is now shown his default
inbox in the \emph{index} view. Here he sees a list of messages, some of which
are marked 'new'. The status bar tells him there are other folders which
contain new messages.

Jack decides to first read the new mail in his inbox. This opens up the
\emph{pager} view. Here he sees a small selection of headers, followed by
the body of the e-mail. He scrolls through and decides he wishes to reply.
So he presses the reply button. This selects the identity the mail was sent
to and matching signature and opens the default editor on a message
containing just the signature. After typing his e-mail, he quits the editor
and is shown the \emph{compose} screen. Here he can make some adjustments
with regard to the subject of the e-mail and some optional other recipients,
and optionally some settings provided by plugins.

Now, he decides to e-mail his boss about that possible raise he was hoping
for, so he presses the 'compose mail' button. He then gets the opportunity to
choose an id in the \emph{id} view, where he selects his 'Jack at work' id.
The matching signature is a mandatory disclaimer from his company and the PGP
key is the one matching his address at work. Now, he enters the
\emph{address book} view, where he selects his boss' address. After typing
his mail, he starts checking the mailing lists he is subscribed to.

To do so, he opens up the \emph{browser} view. This shows his mailfolder,
with the mailboxes and subdirectories. If a mailbox is marked 'new', there
are new (or unread) messages in the box. If a subdirectory is marked 'new',
there is at least one mailbox underneath which contains new messages.
Selecting the list, he comes back to the \emph{index}, where the inbox is
replaced by the list's e-mail messages. The messages are displayed in
threaded mode, because Jack supplied this as the mode for this mailbox.
He also supplied in the configuration that he wants all threads to be
collapsed unless there are unread messages somewhere in the thread.
He quickly skims the messages, marking those which appear not to be of
interest to him as read. The statusbar now changes to indicate there are no
new messages, including in other mailboxes, so Jack decides to finally get
to work and quits Hooch.


\chapter{The views}

\section{Index}

The default screen shown at startup. This view shows mails in the currently
selected mailbox, optionally threaded by topic. It shows a user-customisable
string for each e-mail in the list, which can contain any combination of
headers in an e-mail. Each listed entry also shows a number of flags, which
are extendable by plugins. These flags show the status of the particular
message, which can indicate for example if the message has already been read
or not, whether the message is encrypted or signed etc.
Messages can also be color coded or in some other way formatted, depending on
which interface is used. 

The index shows in the status bar how many messages are in the currently
selected mailbox, the number of flagged messages and the name of the mailbox.
The user can optionally format the layout of the statusbar.
By selecting a message, it can be opened. This will open the internal pager by
default, but the message can also be passed on to another program. It is
always possible to pass the message on to another program, regardless of the
default operation.
Most operations which are available from the pager are also available in the
index, for when the user wishes to open messages with a specific program
these operations must still be available. See pager for a description of these
operations.


\section{Pager}

The pager is the view which the user will see when he has selected an e-mail
to view. It provides functionality to scroll through it, and


\section{Address Book}

The screen greeting the user when he enters the address book, is a list of
all groups, people and lists (collectively called 'aliases') which are not
entered in any subgroup. When an entire group is added, all default e-mail
addresses of the members of this group are added to the currently active
field. Alternatively, the address can be entered into a different field,
when the corresponding button is pressed.
\\XXX FIXME: All default addresses? Better select which addresses belong to
which group and have that selected!\\
Also, a group can be 'opened', which replaces the previous list by a list of
persons and subgroups in the group.
If a person's address is added, the user can do one of two things: Either
'select' the person, which adds the default e-mail address to the current
field, or 'open' the person as if he or she were a group, and select the
desired address. When aliases are entered into an address field, the item in
the view becomes marked, indicating whether all or some of the members of
a group are entered, whether the default or another address of a person
is entered or when a list is entered.

Aliases can be added or deleted from the address book view. When creating
a new alias, the item is automatically inserted in the currently selected
group (if no group is selected, the alias will not be entered in any group).
If a person is created, the user gets the \emph{person} form where
information about a person can be entered. The same holds for a list, opening
the \emph{list} form. If a group is created, the user can select aliases
which are to be entered as members of the group in the same fashion as
aliases are added to an address field when composing mail.
\\XXX TODO: Also here, how to deal with the case of specifically selected
addresses?

\subsection{List form}

\subsection{Person form}


\section{Id}


\section{Browser}


\section{Compose}


\chapter{Global functionality}

There is functionality which is available globally, which means it is
available everywhere except where it is appropriate. This functionality is
described in this chapter.

\section{Searching, tagging and limiting}

One can perform searching anywhere where Hooch is not expecting any input
in the form of direct text.
Searching is done by pressing the search button and then typing a
pattern which follows the same syntax as Mutt's patterns.
When a search pattern has been entered, the first hit down from the current
position is shown. The next can be shown by pressing the 'next match' key.
One can also search backwards.

Tagging works like searching, except that \emph{all} matches are tagged at
once. After that, operations can be performed on all tagged
entries by first typing the 'apply to tagged' button and then the desired
operation as it would be applied on a single entry. Tags are not remembered
across mailboxes.

Limiting works like tagging, but now only the matches are shown while all
other entries are hidden. Limits are removed when changing mailboxes.

Tagging and limiting do not work in the pager, searching does.


\section{Completion}

When Hooch is expecting direct text input, pressing a completion key (usually
tab) will try to finish the currently typed word. If there is only one match,
the match will be completed, followed by what the user will most likely type
directly following the current entry. In address fields for example, a comma
and a space will be added after the current alias. If the current word can
be finished partially with certainty, but after that there are multiple
options, the word will be finished partially. If there are multiple options
from the currently typed word (including the empty word), the appropriate
view is opened containing only those entries that match the partial word
already entered.


\end{document}
