%
% $Id $
%
% Quickly hacked together, may contain lots of too-long descriptions,
% mistakes etc.  I just wanted to jot this down before we forgot it.
%

\documentclass[a4paper]{article}

\title{Design rationale for Turner and Hooch}
\date{May 2004}
\author{Peter~Bex\and{}Vincent~Driessen}

\begin{document}

\maketitle

\section{General configuration files syntax}

The syntax of all Hooch and Turner config files is the same.
The basic notation is borrowed from C-like languages and inspired by
Irssi's configuration files and CSS notation.  This should make it easy
to read for experienced programmers.  Beginners should not need to edit
most config files manually, since the user interface shall at some point
support some way of setting preferences, so there was no reason to worry
about the somewhat cryptic nature (for beginners, anyway) of C.

For ease of parsing, a semicolon is required at the end of a statement.
This allows the user to make statements span multiple lines in the file,
providing maximum flexibility of notation.  Whitespace and comments are
also completely ignored for this reason.

For consistency with many other Unix tools, a hash symbol (\#) marks the
beginning of a comment that runs until the end of a line.


\section{Contact list design rationale}

Contacts in many other Unix mail clients are just simple aliases for a
name and an email address.  Because it is often the case that one
person has multiple email addresses (like one for home, one for work),
Hooch binds these addresses under one contact to keep them together and
perform operations on the basis of contacts instead of on email
addresses.

The system used for this is somewhat like CSS.  Settings that belong to a
certain contact are stored in the contact's main id and optionally
overridden in a sub-id.  By doing so, we can allow the main id's settings
to also pertain to sub-ids of the contact.  When we mail to another email
address of the same person, the name of that person generally does not
change, neither as do many other properties.  These are properties of
the \emph{contact}, not of the alias/email address.

On the other hand, there are certain things that \emph{sometimes} differ
from email address to email address, like a PGP key, for example, or
perhaps the ability to receive HTML mail.  It does not matter which of
the options these are, that cannot be predicted in advance.  Also, plugins
might define new options which might or might not be carried over into
sub-ids.


\subsection{Groups}

Groups are a very useful feature, since users often want to mail to
several people interested in a subject (say students attending a course)
and do not want to manually select these people from a large address
book every time.

Treating groups as simply a collection of e-mail addresses is a bad idea
because Hooch supports the notion of contacts.  If one modifies the
address of a contact, one should not also have to modify this address in
all groups that address is in as well.

Simply wrapping a `group' keyword around contacts is not acceptable, for
two reasons:
\begin{itemize}
\item Contacts can be members of more than one group
\item When a contact is a member of two groups, we might want to use
	their business or their personal id, depending on which group we
	mail to.
\end{itemize}
For this last reason, a new notation had to be decided, which is used
subsequently in other config files as well.  This is the colon
operator.  When we wish to mail to homer's work address, we say
\texttt{homer:work}.  We cannot use the notation \texttt{homer.work} as
this would imply the value of the \texttt{work} option of \texttt{homer}.

For groups, the settings of a contact might conflict with other members
of the group.  For example, when working in a project group, one of the
members may not want to receive, HTML encoded mail.  If this is the
case (and we are using a HTML composer), we can set the option that
enables HTML attachments to \texttt{false} in the group, while allowing
us to send HTML mail to those who prefer it when we mail them personally
or from another group.

Thus, settings in a group override those of the contacts (or actually,
ids of contacts) in those groups.  Because, again, we cannot predict
which settings make sense (for example, setting a group-wide PGP key ID for
encryption probably isn't a very good idea, but who knows?  Perhaps some
group one day decides they share a PGP key), we should not make any
assumptions about which options may be overridden by a group.

All options that are legal inside a contact (or id) are also legal
inside a group.  If it makes sense is up to the user.

\end{document}
