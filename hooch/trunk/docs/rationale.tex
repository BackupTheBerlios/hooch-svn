%
% $Id$
%
% Quickly hacked together, may contain lots of too-long descriptions,
% mistakes etc.  I just wanted to jot this down before we forgot it.
%

\documentclass[a4paper]{article}

\title{Design rationale for Turner and Hooch}
\date{May 2004}
\author{Peter~Bex\and{}Vincent~Driessen}

\begin{document}

\maketitle

\section{General configuration files syntax}

The syntax of all Hooch and Turner config files is generally speaking the same.
The basic notation is borrowed from C-like languages and inspired by
Irssi's configuration files and CSS notation.  This should make it easy
to read for experienced programmers.  Beginners should not need to edit
most config files manually, since the user interface shall eventually
support some way of setting preferences, so there was no reason to worry
about the somewhat cryptic nature (for beginners, anyway) of C.

There is a C struct-like notation to group options that belong together,
like all options that are applicable to a certain plug-in.  This is to
bring some structure into the config files.  This avoids the mess that
many programs' config files are, especially when Hooch's feature set grows
as the program matures and new plug-ins are installed.
Because the user does not always fully utilise every option a plug-in or
subsystem has to offer, the dot notation can be used for compactness.

For ease of parsing, a semicolon is required at the end of a statement.
This allows the user to make statements span multiple lines in the file,
providing maximum flexibility of notation.  Whitespace and comments are
also completely ignored for this reason.  Because a closing brace (\}) is
always the end of a list of assignments to one structure's fields, there
is no need for a semicolon after a closing brace.  Thus, to avoid confusion,
the semicolon is explicitly disallowed after closing braces.

For consistency with many other Unix tools, a hash symbol (\#) marks the
beginning of a comment that runs until the end of a line.


\section{Contact list design rationale}

Contacts in many other Unix mail clients are just simple aliases for a
name and an email address.  Because it is often the case that one
person has multiple email addresses (like one for home, one for work),
Hooch binds these addresses under one \emph{contact} to keep them together
and perform operations on the basis of contacts instead of on email
addresses.

The system used for this is somewhat like CSS.  Settings that belong to a
certain contact are stored in the contact's main id and optionally
overridden in a sub-id.  By doing so, we can allow the main id's settings
to also pertain to sub-ids of the contact.  When we mail to another email
address of the same person, the name of that person generally does not
change, neither as do many other properties.  These are properties of
the \emph{contact}, not of the alias/email address.

On the other hand, there are certain things that \emph{sometimes} differ
between email addresses of one person, like a PGP key for example, or
perhaps the ability to view HTML mail at that address' site.  It does not
matter which of the options these are, that cannot be predicted in advance.
Also, plugins might define new options which might or might not be carried
over into sub-ids.  For this reason, every option of a contact may be
overridden in a sub-id, even if this would not make much sense conceptually.

Requiring each id to be a sub-id (ie, not allowing an \texttt{address}
field in the main body of the contact) would be irritating in the simple
case of a pure name/address pair with nothing else added.  Also, when
mailing to a contact, most of the time we do not bother about some
secondary address, we just wish to send mail to that person's primary
address.  Not having to specify a sub-id is more practical then.


\subsection{Groups}

Groups are a very useful feature, since users often want to mail to
several people interested in a subject (say, students attending a course)
and do not want to manually select these people from a large address
book every time.

Treating groups as simply a collection of e-mail addresses is a bad idea
because Hooch supports the notion of contacts.  If one modifies the
address of a contact, one should not also have to modify this address in
all groups that address is in as well.

Having a \texttt{groups} option in a contact listing the groups the
contact is a member of is a possible solution, but it is not very clear.
How can one get a simple overview of all the contacts in a group?  One would
have to walk through the entire file to find them all.  As can be read
below, our solution to the groups problem is more flexible, as well.

Simply wrapping a \texttt{group \{ .. \}} block around contacts is not
acceptable, for two reasons:
\begin{itemize}
\item Contacts can be members of more than one group.
\item When a contact is a member of two groups, we might want to use
	their business or their personal id, depending on which group we
	mail to.
\end{itemize}
For this last reason, a new notation had to be decided, which is used
subsequently in other config files as well.  This is the colon
operator.  When we wish to mail to Homer's work address, we say
\texttt{homer:work}.  We cannot use the notation \texttt{homer.work} as
this would imply the value of the \texttt{work} option of \texttt{homer}.

For groups, the settings of a contact might conflict with other members
of the group.  For example, when working in a project group, one of the
members may not want to receive, HTML encoded mail.  If this is the
case (and we are using a HTML composer), we can set the option that
enables HTML attachments to \texttt{false} in the group, while allowing
us to send HTML mail to those who prefer it when we mail them personally
or from another group.

Thus, settings in a group override those of the contacts (or actually,
ids of contacts) in those groups.  Because, again, we cannot predict
which settings make sense (for example, setting a group-wide PGP key ID for
encryption probably isn't a very good idea, but who knows?  Perhaps some
group one day decides they want to share a PGP key), we should not make any
assumptions about which options may be overridden by a group.
For this reason, all options that are legal inside a contact (or id) are
also legal inside a group.  If it makes sense is up to the user.

\end{document}
