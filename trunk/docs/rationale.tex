%
% $Id$
%
% Quickly hacked together, may contain lots of too-long descriptions,
% mistakes etc.  I just wanted to jot this down before we forgot it.
%
\documentclass[a4paper]{article}

\title{Design rationale for Turner and Hooch}
\date{May 2004}
\author{Peter~Bex\and{}Vincent~Driessen}

\begin{document}

\maketitle

\section{Syntax}

\subsection{Origin}

The syntax of all Hooch and Turner config files is generally speaking
the same.  The basic notation is borrowed from C-like languages and
inspired by Irssi's configuration files and CSS notation.  This should
make it easy to read for experienced programmers.  Beginners should not
need to edit most config files manually, since the user interface shall
eventually support some way of setting preferences, so there was no
reason to worry about the somewhat cryptic nature (for beginners,
anyway) of C.
 
\subsection{Grouping options}

There is a C struct-like notation to group options that belong together,
like all options that are applicable to a certain plug-in.  This is to
bring some structure into the config files.  This avoids the mess that
many programs' config files are, especially when Hooch's feature set
grows as the program matures and new plug-ins are installed.  Because
the user does not always fully utilise every option a plug-in or
subsystem has to offer, the dot notation (\texttt{struct.option =
value;}) can be used for compactness.

\subsection{Statements}

For ease of parsing, a semicolon is required at the end of a statement.
This allows the user to make statements span multiple lines in the file,
providing maximum flexibility of notation.  Whitespace and comments are
also completely ignored for this reason.  Because a closing brace
(\verb.}.) is always the end of a list of assignments to one structure's
fields, there is no need for a semicolon after a closing brace.  Thus,
to avoid confusion, the semicolon is explicitly disallowed after closing
braces.

\subsection{Comments}

For consistency with many other Unix tools, a hash symbol (\#) marks the
beginning of a comment that runs until the end of a line.

\subsection{\texttt{empty}}

Because in some situations options are inherited from structures higher
up in the hierarchy (CSS-like), there has to be a way to unset some
options which are set in the parent structure.  For this reason, the
\texttt{empty} keyword was introduced.  This resets the field for this
substructure to what it would be if it or its parent had never been set
in the first place.

The clearest example is probably the PGP key ID which is set on some
contact's primary email address, while at the secondary's address site
PGP is not available.  Here, the secondary address' PGP key ID value
should be unset to prevent Hooch's PGP plugin from signing an email.


\part{Hooch}

\section{Contact list design rationale}

\subsection{Identities}

Contacts in many other Unix mail clients are just simple aliases for a
name and an email address.  Because it is often the case that one person
has multiple email addresses (like one for home, one for work), Hooch
binds these addresses under one \emph{contact} to keep them together and
perform operations on the basis of contacts instead of on email
addresses.  These addresses are called contact identities, or
\emph{identities}, for short.

We used a cascading properties system for this.  Settings that belong to
a certain contact are stored in the contact's main idenity and
optionally overridden in a sub-identity.  By doing so, we can allow the
main identity's settings to also pertain to sub-identities of the
contact.  When we mail to another email address of the same person, the
name of that person generally does not change, neither as do many other
properties.

On the other hand, there are certain things that \emph{sometimes} differ
between email addresses of one person, like a PGP key for example, or
perhaps the ability to view HTML mail at that address' site.  It does
not matter which of the options these are---that cannot be predicted in
advance.  Also, future plugins might define new options which might or
might not be carried over into sub-identities.  For this reason, every
option of a contact may be overridden in a sub-identity, even if this
would not make much sense conceptually.

Requiring each identity to be a sub-identity (i.e., not allowing an
\texttt{address} field in the main body of the contact) would be
irritating in the simple cases that a contact has no need for a
sub-identity (i.e. in most cases).  Also, when mailing to a contact,
most of the time we do not bother about some secondary address, we just
wish to send mail to that person's primary address.  Not having to
specify a sub-identity is more practical then.  Therefore we introduced
some syntactical sugar for this, allowing short primary identity
declarations:

An example:
\begin{verbatim}
# This...
contact jose {
    primary {	# "primary" is a mandatory identity
        name = "Jose";
        address = "jose@hola.es";
    }
}

# ...can be written in short as:
contact jose {
    name = "Jose";
    address = "jose@hola.es";
}
\end{verbatim}

Note that only one of these two types of \texttt{primary} identity
declarations are allowed per contact.  For example, the contact declaration
below would be a false one, and should result in a parsing error:

\begin{verbatim}
contact jose {
    name = "Jose";
    address = "jose@hola.es";

    primary {	# False: there are primary fields already
        name = "Jose";
        address = "jose@hola.es";
    }
}
\end{verbatim}


\subsection{Defaults}

\subsubsection{Rejected ideas}

Because it is not practical to specify many options common to each
contact every time a new contact is defined, there should be a way to
define default options applicable to all contacts.  We considered
defining a \textbf{special contact} called `default', but this is not a
good idea because this contact should not contain identities or an
address field.  The `cascade' idea of properties which fall down from
the default to a contact to an identity does not apply here either.
After all, defaults are not a contact at all.

Putting the defaults among the contacts (intermixing them with contacts)
is another option, but would not be consistent with the idea of grouping
relevant settings together and result in a mess.

Putting defaults in the \texttt{hoochrc} file is not a solution either,
because this would also separate the settings relevant to contacts into
two files, and if these settings might be useful to Turner (or another
program), this is not desirable.

\subsubsection{Accepted idea}

The current solution is to allow exactly one \texttt{defaults} block
which may contain only the behaviour settings of identities.

\subsubsection{Overridable settings}

Not all settings may be defaulted in the \texttt{default} block.  There
are two types of options that can be set.  For the end user, they are no
different, but internally, we make the distinction in order to overcome
conceptually wrong contructs in the contacts file.  The two types of
settings are \textbf{informational} and \textbf{behavioural} ones.

Informational settings belong to contacts or identities and are
conceptually wrong to default anywhere.  For example, defaulting a
contact's phone number would be pointless, for these properties always
differ per user, in principle.

Behavioural settings, on the contrary, are very good candidates to
default.  Say one wants to send out HTML mail per default.  In these
cases, one could default the (imaginary) setting
\texttt{enable\_html\_attachments} to \texttt{yes}, and overwrite the
setting in all specific contacts that do not favor HTML mails.

Thus, note that there is no technical diffence between the two types of
settings, but a distinction is merely made to avoid conceptually wrong
constructs.


\subsection{Groups}

Groups are a very useful feature, since users often want to mail to
several people interested in a subject (say, students attending a
course) and do not want to manually select these people from a large
address book every time.

\subsubsection{Rejected ideas}

Treating groups as simply a collection of e-mail addresses is a bad idea
because Hooch supports the notion of contacts.  If one modifies the
address of a contact, one should not also have to modify this address in
all groups that address is in as well.

Having a \texttt{groups} option in a contact, summarizing the groups the
contact is a member of, is a possible solution, but it is not very
clear.  How can one get a simple overview of all the contacts in a
group?  One would have to walk through the entire file to find them all.
As can be read below, our solution to the groups problem is more
flexible, as well.

Simply wrapping a \verb=group { ... }= block around contacts is not
acceptable, for two reasons:

\begin{itemize}
\item Contacts can be members of more than one group.
\item When a contact is a member of two groups, we might want to use
      their business or their personal id, depending on which group we
      mail to.
\end{itemize}

\subsubsection{Accepted idea}

For this last reason, a new notation had to be decided, which is used
subsequently in other config files as well.  This is the colon operator.
When we wish to mail to Homer's work address, we say
\texttt{homer:work}.  We cannot use the notation \texttt{homer.work} as
this would refer to the value of the \texttt{work} option of
the contact \texttt{homer}.

Another advantage of the new idea is that when mailing to a certain group
one might want to set different options, for example use a disclaimer
signature file that was specifically created for mailings to this group.

An example:
\begin{verbatim}
group workers {
	signature = ~/Mail/disclaimer.sig;
	members = homer:work, smithers:home;
}
\end{verbatim}

This introduces a new kind of data type, namely the array.  Because
groups can contain many members, it is nice to be able to split them
across multiple lines.  For this, one could of course omit the semicolon
and just continue on the next line.  But it would be nicer to be able
to split these long lists into multiple lines.  For this, we introduced
the \verb|+=| operator.

An example:
\begin{verbatim}
group kids {
	members =  bart, lisa;
	members += maggy;
}
\end{verbatim}


\subsubsection{Behaviour when settings conflict}

For groups, the settings of a contact might conflict with other members
of the group.  For example, when working in a project group, one of the
members may not want to receive, HTML encoded mail.  If this is the
case (and we are using a HTML composer), we can set the option that
enables HTML attachments to \texttt{false} in the group, while allowing
us to send HTML mail to those who prefer it when we mail them personally
or from another group.

Thus, settings in a group override those of the contacts (or actually,
ids of contacts) in those groups.  Because, again, we cannot predict
which settings make sense (for example, setting a group-wide PGP key ID for
encryption probably isn't a very good idea, but who knows?  Perhaps some
group one day decides they want to share a PGP key), we should not make any
assumptions about which options may be overridden by a group.
For this reason, all behaviour settings that are legal in identities are
also legal inside a group.  If it actually makes sense is up to the user.

%
% XXX TODO FIXME: We still haven't defined what the hell should happen in
% case of conflicting settings.
%
% I vote for treating every conflicted setting as if it was set to `empty',
%  thus picking the default in case of conflict.  Of course, if the group
%  overrides it we can simply use that setting.  Maybe... maybe we should
%  not even allow settings to be taken from group members.  That might
%  make things easier and more predictable from the user's POV.  On the
%  other hand, it is rather cool to have an e-mail client that's intelligent
%  enough to understand that all contacts in a group think HTML
%  attachments are a Good Thing, thus setting that option to true.
%

\part{Turner}

% XXX: Discuss the syntax we dropped in revision 40 of turner.sample.
\section{Action script syntax}


\section{The problem with headers}

In traditional mail processors, one uses regular expressions on the lines in an
e-mail directly, without referring to the e-mail addresses in a \texttt{To}-field,
for example.  This is too low-level for most practical situations, where one would
rather like to refer to collections of e-mail addresses of \emph{contacts}.

For this reason, we introduced two keywords, \texttt{contains} and
\texttt{matches}, which mean the header field should be interpreted as a list
of e-mail addresses or just as a string which can be matched against a regular
expression, respectively.


\end{document}
